\documentclass[12pt]{article}

\usepackage{psfig}

\newcommand{\bcenter}{\begin{center}}
\newcommand{\ecenter}{\end{center}}
\newcommand{\bitem}{\begin{itemize}}
\newcommand{\eitem}{\end{itemize}}
\newcommand{\bdesc}{\begin{description}}
\newcommand{\edesc}{\end{description}}
\newcommand{\benum}{\begin{enumerate}}
\newcommand{\eenum}{\end{enumerate}}

\begin{document}

\bigskip

\bcenter\bf\Large
Hall A Parity Database Specification \\
\bigskip
\large R.~Michaels and R.~Suleiman \\
\medskip
November 21, 2001
\ecenter

\section{Introduction}

This document is a programming specification for Hall A Parity
database. The database contains logical parameters that affect
program flow, the datamap correspondence of 'keys' to the
event buffer, calibration parameters used by the program, and
cut definitions. The values selected from the database
will in general depend on the run number of the data being
analyzed.

The database is implemented in MySQL \cite{MySQL}. There is 
C++ class called TaMysql written in ROOT \cite{root} to 
interface with the database.

\section{Features}
\bitem

\item  For the "Pan" code, the database contains:

\bitem
     \item[1.] Logical Parameters that affect program flow.
     \item[2.] DataMap (correspondence of 'keys' to event buffer).
     \item[3.] Calibration Parameters used by program, e.g.
          DAC noise calibration slope and intercept
          ADC pedestal
    \item[4.] Cuts and the event intervals associated.
        Cuts include 'lobeam', 'bcm1', 'paircheck', 'doc',
        'beamburp' etc (as in HAPPEX I).  
\eitem

\item Run indexed lookup of constants by analysis jobs.

\item Keep a change log: user, date, time, comment.

\eitem

\section{Scripts}

There will be a script to generate ascii files from the database, and one
to do the reverse, generate a database from the file. 



\small\begin{verbatim}

 ascii2db.pl 

 db2ascii.pl run=100

\end{verbatim}\normalsize


\section{Database Tables}

The database tables are shown schematically in Fig.~\ref{schematic}
and are listed explicitly in Tables~\ref{infotables}-\ref{databasetables}.
The perl script ``dbcreate.pl'' will create the database.
\begin{figure}
$$
\psfig{figure=dbdiag.eps,width=5.0in,clip=t}
$$
\caption{Schematic drawing of the database tables.}\label{schematic}
\end{figure}


\begin{table}
$$
\begin{tabular}{|l|l|l|l|} \hline
\multicolumn{4}{|c|}{RunInfo} \\
\hline
column name & type & example & comment \\
\hline\hline
RunNumber   & int        & 2000        & Primary key \\
Experiment  &  varchar   &  ``HAPPEXII'' &  \\
time        & datetime   &  2003-12-09 11:30:45&  \\
comment     & text       &             &  \\
\hline
\end{tabular}
$$


$$
\begin{tabular}{|l|l|l|l|} \hline
\multicolumn{4}{|c|}{System} \\
\hline
column name & type & example & comment \\
\hline\hline
systemId    & int     & 1              & Primary key auto increment\\
systemName  & varchar & ``dacnoise'' &  \\
description & text    &                & \\
\hline
\end{tabular}
$$


$$
\begin{tabular}{|l|l|l|l|} \hline
\multicolumn{4}{|c|}{SubSystem} \\
\hline
column name & type & example & comment \\
\hline\hline
subsystemId    & int     &  1           & Primary key auto increment\\
subsystemName  & varchar & ``adc0'' & \\
systemId       & int     &  1           & Foreign key reference System\\
description    & text    &              & \\
\hline
\end{tabular}
$$


$$
\begin{tabular}{|l|l|l|l|} \hline
\multicolumn{4}{|c|}{Item} \\
\hline
column name & type & example & comment \\
\hline\hline
itemId      & int     & 1         & Primary key auto increment\\
itemName    & varchar & ``chan0''   & \\
subsystemId & int     & 1         & Foreign key reference SubSystem\\
length      & int     & 2         & Number of elements\\
type        & varchar & ``float'' & \\
description & text    &           & \\
\hline
\end{tabular}
$$
\caption{Informational tables.}\label{infotables}
\end{table}



\begin{table}
$$
\begin{tabular}{|l|l|l|l|} \hline
\multicolumn{4}{|c|}{RunIndex} \\
\hline
column name & type & example & comment\\
\hline\hline
RunIndexId     & int        & 1       & Primary key auto increment\\
RunNumber      & int        & 2000   &  \\
itemId         & int        & 1       & Foreign key reference Item\\
itemValueId    & int        & 1       & Foreign key reference item\_name\_VALUE\\
author         & varchar     & ``dbmanager''            & \\
time           & timestamp  & 20000502165717        & \\ 
comment        & text       &                       & \\
\hline
\end{tabular}
$$


$$
\begin{tabular}{|l|l|l|l|} \hline
\multicolumn{4}{|c|}{dacnoise\_adc0\_chan0\_VALUE} \\
\hline
column name & type & example & comment\\
\hline\hline
itemValueId    & int        &  1       & Primary key auto increment\\
value\_1       & ``float''   &  1.82    &  \\
author         & varchar    & ``smith''& \\
time           & timestamp  & 20000502165717 & \\ 
comment        & text       &                & \\
\hline
\end{tabular}
$$

\caption{Functional tables.}\label{databasetables}
\end{table}


%%%


Notes on this structure:

\bdesc

\item[Values are ``never'' deleted from the database.] 

\edesc

\section{Access Privileges}

There are several different levels of access to the database:
\benum
\item database manager
\item users
\eenum

The MySQL access tables shown in Tables
\ref{usertable}-\ref{dbtable}. 
The perl script ``access\_priv.pl'' creates these tables. 

\begin{table}
$$
\begin{tabular}{|l|l|l|l|} \hline
\multicolumn{4}{|c|}{user table}\\
\hline
      Host                & User      & Password & Privileges \\
\hline\hline
      localhost           & dbmanager & non-NULL & none \\
      \%                  & dbuser      & NULL     & none \\
\hline
\end{tabular}
$$
\caption{MySQL user table.
Any user from any host can connect to the database.}
\label{usertable}
\end{table}

\begin{table}
$$
\begin{tabular}{|l|l|l|l|}
\hline
\multicolumn{4}{|c|}{db table}\\
\hline
      Host  & Db     &User      &Privileges\\
\hline\hline
       \%   &  pandb & dbmanager& all\\
       \%   &  pandb & \%       &  select\\
\hline
\end{tabular}
$$
\caption{MySQL db table. The dbmanager has all privileges in the pandb
database. Any user that can connect can read any of the tables in the
pandb database.}
\label{dbtable}
\end{table}


\begin{thebibliography}{9}

\bibitem{MySQL} http://www.mysql.org

\bibitem{root} http://root.cern.ch

\end{thebibliography}


\end{document}
